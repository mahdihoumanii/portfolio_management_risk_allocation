\documentclass[11pt,a4paper]{article}

\usepackage[margin=1in]{geometry}
\usepackage{graphicx}
\usepackage{amsmath}
\usepackage{booktabs}
\usepackage{hyperref}
\usepackage{float}
\usepackage{parskip}

\title{\textbf{Portfolio Management and Risk Allocation}\\
\large Walk-Forward Backtesting of Allocation Strategies}
\author{Mohamad ElMahdi Houmani}
\date{\today}

\begin{document}
\maketitle

\section{Introduction}

This project implements a full portfolio management and backtesting framework to study classical and risk-based asset allocation strategies under realistic market assumptions.  
The objective is to compare the risk–return characteristics, turnover, and drawdown behavior of different allocation approaches using a consistent walk-forward methodology with transaction costs.

The framework is designed to be fully reproducible and modular, with cached data loading, rolling covariance estimation, constrained optimization, and comprehensive performance evaluation.

\section{Data and Returns}

The investment universe consists of liquid exchange-traded funds representing multiple asset classes:

\begin{itemize}
    \item SPY (US equities)
    \item QQQ (US technology equities)
    \item EEM (emerging market equities)
    \item TLT (long-duration US Treasuries)
    \item IEF (intermediate Treasuries)
    \item GLD (gold)
    \item VNQ (US real estate)
\end{itemize}

Daily adjusted closing prices are downloaded via \texttt{yfinance} and cached locally.  
Log returns are computed as
\[
r_t = \log\left(\frac{P_t}{P_{t-1}}\right),
\]
and all annualized statistics use 252 trading days per year.

\section{Covariance Estimation}

Portfolio construction relies on rolling estimates of the return covariance matrix using a 252-day lookback window.  
The following estimators are implemented:

\begin{itemize}
    \item \textbf{Sample covariance}
    \item \textbf{EWMA covariance} with decay parameter $\lambda = 0.94$
    \item \textbf{Shrinkage covariance} toward the diagonal to reduce estimation noise
\end{itemize}

Shrinkage improves numerical stability and reduces excessive portfolio turnover, particularly in constrained optimization problems.

\section{Portfolio Strategies}

All strategies are long-only and fully invested unless otherwise stated.

\subsection{Equal Weight}
A baseline strategy allocating equal capital to each asset.

\subsection{Minimum Variance}
Weights are chosen to minimize portfolio variance:
\[
\min_w \; w^\top \Sigma w \quad \text{s.t.} \quad \sum_i w_i = 1,\; w_i \ge 0.
\]

\subsection{Mean--Variance}
A classical Markowitz formulation:
\[
\max_w \; \mu^\top w - \frac{\gamma}{2} w^\top \Sigma w,
\]
where expected returns $\mu$ are estimated using rolling historical means and $\gamma$ controls risk aversion.

\subsection{Risk Parity}
Weights are chosen such that each asset contributes equally to total portfolio risk.  
An iterative solver enforces equal risk contributions under long-only constraints.

\subsection{Volatility Targeting}
A volatility targeting overlay rescales portfolio exposure to target a fixed annualized volatility (10\%), with leverage capped at 1.5.  
Any remaining allocation is assigned to cash (assumed zero return).

\section{Backtesting Methodology}

A walk-forward backtest is performed with monthly rebalancing.  
At each rebalance date:
\begin{itemize}
    \item Covariance and expected returns are estimated using only past data.
    \item Portfolio weights are recomputed.
    \item Transaction costs are applied based on turnover:
    \[
    \text{Cost}_t = c \sum_i |w_i^{(t)} - w_i^{(t-1)}|,
    \]
    where $c = 5$ basis points.
\end{itemize}

Daily portfolio returns are computed using lagged weights to avoid look-ahead bias.

\section{Performance Metrics}

For each strategy, the following metrics are computed:
\begin{itemize}
    \item CAGR
    \item Annualized volatility
    \item Sharpe and Sortino ratios
    \item Maximum drawdown
    \item Calmar ratio
    \item Average turnover
    \item Value-at-Risk (95\%) and Conditional VaR (95\%)
\end{itemize}

The full metrics table is saved as \texttt{reports/metrics.csv} and loaded by both the CLI and the report.

\section{Results}

\subsection{Cumulative Performance}

\begin{figure}[H]
\centering
\includegraphics[width=0.9\textwidth]{figures/cumulative_returns.png}
\caption{Cumulative portfolio growth for all strategies.}
\end{figure}

Mean--variance allocation achieves the highest long-term growth, while risk parity exhibits lower returns and larger drawdowns in this universe.

\subsection{Rolling Volatility}

\begin{figure}[H]
\centering
\includegraphics[width=0.9\textwidth]{figures/rolling_vol.png}
\caption{63-day rolling annualized volatility.}
\end{figure}

Volatility spikes during the 2020 crisis are clearly visible.  
The volatility targeting strategy successfully caps realized risk relative to other approaches.

\subsection{Turnover}

\begin{figure}[H]
\centering
\includegraphics[width=0.9\textwidth]{figures/turnover.png}
\caption{Portfolio turnover by strategy.}
\end{figure}

Risk parity and mean--variance strategies exhibit higher turnover, explaining their sensitivity to transaction costs.

\subsection{Portfolio Weights}

\begin{figure}[H]
\centering
\includegraphics[width=0.9\textwidth]{figures/weights_min_var.png}
\caption{Minimum variance portfolio weights.}
\end{figure}

The minimum variance portfolio closely resembles the equal-weight benchmark due to strong cross-asset correlations and covariance shrinkage.

\begin{figure}[H]
\centering
\includegraphics[width=0.9\textwidth]{figures/weights_risk_parity.png}
\caption{Risk parity portfolio weights.}
\end{figure}

Risk parity allocations are highly dynamic, especially during periods of elevated volatility.

\section{Discussion}

The results highlight several important insights:
\begin{itemize}
    \item Covariance shrinkage stabilizes optimization but can collapse minimum variance solutions toward equal weighting.
    \item Mean--variance allocation improves returns at the cost of higher turnover.
    \item Risk parity without leverage may underperform in equity-dominated regimes.
    \item Volatility targeting provides effective downside risk control.
\end{itemize}

\section{Limitations}

The framework assumes:
\begin{itemize}
    \item No short-selling
    \item Constant transaction costs
    \item No estimation uncertainty beyond rolling windows
\end{itemize}

Extensions could include regime detection, Bayesian return estimation, or derivative-based risk overlays.

\section{Conclusion}

This project demonstrates a complete, production-style portfolio management pipeline, combining sound financial theory with robust numerical implementation and realistic backtesting discipline.

\end{document}